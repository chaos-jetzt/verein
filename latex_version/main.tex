\documentclass[11pt,a4paper]{scrartcl}

\usepackage{fontspec}
\setmainfont{Linux Libertine O}

\usepackage[ngerman]{babel}
\usepackage{amsmath}
\usepackage{amssymb}
\usepackage{csquotes}

% Für detaillierte Listenanpassungen
\usepackage{enumitem}

% Seitenränder
\usepackage[left=2.5cm, right=2.5cm, top=2.5cm, bottom=2.5cm]{geometry}

% Abstand zwischen Absätzen statt Einzug
\usepackage{parskip}
\setlength{\parskip}{1.5ex plus 0.5ex minus 0.2ex}


\begin{document}

\begin{center}
    \Huge\bfseries Vereinssatzung - chaos.jetzt e.V.
\end{center}
\vspace{0.5em}
\begin{center}
    Tag der Errichtung der Satzung: 24.05.2025
\end{center}
\vspace{1.5em}

\section*{§ 1 Name, Sitz, Geschäftsjahr}
\begin{enumerate}[label=\arabic*.]
    \item Der Verein führt den Namen \enquote{chaos.jetzt}. Der Verein wird in das Vereinsregister eingetragen und dann um den Suffix \enquote{e. V.} ergänzt.
    \item Der Verein hat seinen Sitz in Leipzig.
    \item Das Geschäftsjahr ist das Kalenderjahr.
\end{enumerate}

\section*{§ 2 Zweck}
\begin{enumerate}[label=\arabic*.]
    \item Der Verein fördert und unterstützt Vorhaben der Forschung, Wissenschaft, Bildung und Jugendarbeit und führt diese im gesetzlichen Rahmen durch. Der Vereinszweck soll unter anderem durch folgende Mittel erreicht werden:
    \begin{enumerate}[label=\roman*.]
        \item Die Organisation von Veranstaltungen, insbesondere für Jugendliche,
        \item Die Teilnahme an Veranstaltungen im IT-Bereich,
        \item Die Förderung der Kommunikation zwischen technik- und computerinteressierten Jugendlichen,
        \item Die Förderung von praktischen und künstlerischen Projekten Jugendlicher im Bereich der Computertechnik.
    \end{enumerate}
    \item Der Verein fördert und unterstützt Vorhaben, um die Zusammenarbeit zwischen den Mitgliedern zu verbessern.
\end{enumerate}

\section*{§ 3 Mitgliedschaft}
\begin{enumerate}[label=\arabic*.]
    \item Ordentliche Vereinsmitglieder können natürliche Personen, die das fünfundzwanzigste Lebensjahr noch nicht vollendet haben, werden.
    \item Mitglieder, deren Vereinsmitgliedschaft aufgrund satzungsgemäßer Bestimmungen ruht, sind in der Mitgliederversammlung nicht abstimmungs- oder wahlberechtigt, es sei denn es handelt sich um Beschlüsse über den Ausschluss einzelner Mitglieder. Über die Enthebung aus Ämtern entscheidet der Vorstand.
\end{enumerate}

\subsection*{§ 3a Rechte und Pflichten}
\begin{enumerate}[label=\arabic*.]
    \item Die Mitglieder sind verpflichtet, die satzungsgemäßen Zwecke des Vereins zu unterstützen und zu fördern. Sie sind verpflichtet, die festgesetzten Beiträge zu zahlen.
\end{enumerate}

\subsection*{§ 3b Beitritt}
\begin{enumerate}[label=\arabic*.]
    \item Der Aufnahmeantrag erfolgt in Textform gegenüber dem Vorstand.
    \item Die Fördermitgliedschaft steht grundsätzlich jeder natürlichen Person offen.
    \setcounter{enumi}{3} % Springt von 2. auf 4.
    \item Über die Annahme des Aufnahmeantrags entscheidet der Vorstand.
    \item Die Mitgliedschaft beginnt mit der Annahme des Aufnahmeantrags.
    \item Eine Ablehnung des Aufnahmeantrags ist nicht anfechtbar und muss nicht begründet werden.
\end{enumerate}

\subsection*{§ 3c Mitgliedschaftsende}
\begin{enumerate}[label=\arabic*.]
    \item Die Mitgliedschaft endet durch 
    \begin{enumerate}[label=\roman*.]
        \item Austrittserklärung oder
        \item Tod von natürlichen Personen oder 
        \item Ausschluss.
    \end{enumerate}
    \item Der Austritt wird durch Willenserklärung in Textform gegenüber dem Vorstand vollzogen. Der Austritt ist nur mit einer Frist von vier Wochen zum Ende des Geschäftsjahres möglich.
    \item Die Beitragspflicht für das laufende Beitragsjahr bleibt unberührt. Eine Erstattung bereits vereinnahmter Beiträge erfolgt nicht.
    \item Vollendet ein ordentliches Mitglied das siebenundzwanzigste Lebensjahr, endet seine ordentliche Mitgliedschaft mit den verbundenen Rechten und Pflichten. Das ehemalige Mitglied kann daraufhin einen Antrag auf Alumni-Mitgliedschaft stellen.
    \begin{enumerate}[label=\theenumi\arabic*.] % Für 4.1.
        \item Eine Alumni-Mitgliedschaft kommt in ihren Rechten denen einer Fördermitgliedschaft gleich, in ihren Pflichten, insbesondere der Beitragspflicht, jedoch einem ordentlichen Mitglied. Eine erneute Aufnahmegebühr fällt hierbei keinesfalls an.
    \end{enumerate}
    \item Bei Ausscheiden eines Mitgliedes aus dem Verein oder bei Vereinsauflösung erfolgt keine Rückerstattung etwa eingebrachter Vermögenswerte.
\end{enumerate}

\subsection*{§ 3d Ausschluss}
\begin{enumerate}[label=\arabic*.]
    \item Ein Mitglied kann durch Beschluss des Vorstandes ausgeschlossen werden, wenn es
    \begin{enumerate}[label=\roman*.]
        \item das Ansehen des Vereins schädigt, oder
        \item gegen den Ethikkodex verstößt, oder
        \item seinen Beitragsverpflichtungen nicht nachkommt oder
        \item ein sonstiger wichtiger Grund vorliegt.
    \end{enumerate}
    \item Der Vorstand muss dem auszuschließenden Mitglied den Beschluss in Textform unter Angabe von Gründen an die letzte bekannte Anschrift oder an die zuletzt bekannte E-Mail-Adresse mitteilen und ihm auf Verlangen eine Anhörung gewähren.
    \item Der Beschluss des Vorstandes kann vom auszuschließenden Mitglied angefochten werden. Die Anfechtung muss innerhalb einer Frist von vier Wochen ab Zugang des Ausschließungsbeschlusses schriftlich beim Vorstand eingelegt werden. In diesem Fall hat der Vorstand die Mitgliederversammlung anzurufen, welche über den Ausschluss abstimmt. Bis zum Beschluss der Mitgliederversammlung ruht die Mitgliedschaft. Erfolgt innerhalb der Frist keine Anfechtung gilt die Mitgliedschaft ab dem Zeitpunkt des Ausschlusses als beendet.
\end{enumerate}

\subsection*{§ 3e Streichung}
\begin{enumerate}[label=\arabic*.]
    \item Ein Mitglied kann durch Beschluss des Vorstandes von der Mitgliederliste gestrichen werden, wenn es trotz Mahnung mit der Zahlung des Mitgliedsbeitrages länger als ein Jahr im Rückstand ist.
    \item Der Vorstand muss dem betroffenen Mitglied den Beschluss in Textform unter Angabe von Gründen an die letzte bekannte Anschrift oder an die zuletzt bekannte E-Mail-Adresse mitteilen und ihm auf Verlangen eine Anhörung gewähren.
    \item Die Streichung aus der Mitgliederliste erfolgt unabhängig von der Anrufung der Mitgliederversammlung und ist nicht anfechtbar.
\end{enumerate}

\section*{§ 4 Mittel des Vereins}

\begin{enumerate}[label=\arabic*.]
    \item Der Verein erhebt Mitgliedsbeiträge gemäß der Beitragsordnung.
    \item Im Falle nicht fristgerechter Entrichtung der Beiträge ruht die Mitgliedschaft. Dem Vorstand bleibt darüber hinaus der Ausschluss vorbehalten.
\end{enumerate}

\section*{§ 5 Organe des Vereins}
\begin{enumerate}[label=\arabic*.]
    \item Die Organe des Vereins sind
    \begin{enumerate}[label=\roman*.]
        \item die Mitgliederversammlung und
        \item der Vorstand.
    \end{enumerate}
\end{enumerate}

\section*{§ 6 Mitgliederversammlung}
\begin{enumerate}[label=\arabic*.]
    \item Oberstes Beschlussorgan ist die Mitgliederversammlung. Ihrer Beschlussfassung unterliegen:
    \begin{enumerate}[label=\roman*.]
        \item die Wahl und Abberufung der Vorstandsmitglieder,
        \item die Wahl der Kassenprüfenden,
        \item die Entlastung des Vorstandes und der Kassenprüfenden,
        \item die Satzungsänderungen,
        \item die Genehmigung des Finanzberichts,
        \item der Erlass, Änderung und Aufhebung von Ordnungen,
        \item die Anträge des Vorstandes und der Mitglieder, sowie
        \item die Auflösung des Vereins.
    \end{enumerate}
    \item Die Mitgliederversammlung kann an einem durch den Vorstand festgelegten Ort, im Rahmen von Netzkonferenzen zur fernmündlichen Teilnahme sowie in einer Kombination hiervon erfolgen.
\end{enumerate}

\subsection*{§ 6a Einberufung}
\begin{enumerate}[label=\arabic*.]
    \item Die ordentliche Mitgliederversammlung findet alle zwei Jahre statt. 
    \item Außerordentliche Mitgliederversammlungen werden 
    \begin{enumerate}[label=\roman*.]
        \item auf Beschluss des Vorstandes abgehalten, wenn die Interessen des Vereins dies erfordern, oder 
        \item wenn ein Viertel der Mitglieder oder ein Viertel der stimmberechtigten Mitglieder dies unter Angabe des Zwecks schriftlich beantragen. 
    \end{enumerate}
    \item Die Einberufung der Mitgliederversammlung erfolgt in Textform durch den Vorstand mit einer Frist von mindestens zwei Wochen.
    \item Der Einladung ist eine Tagesordnung sowie die Gegenstände der anstehenden Beschlussfassungen beizufügen.
    \item Zur Wahrung der Frist reicht die Aufgabe der Einladung zur Post an die letzte bekannte Anschrift oder die Versendung an die zuletzt bekannte E-Mail-Adresse.
\end{enumerate}

\subsection*{§ 6b Tagesordnung}
\begin{enumerate}[label=\arabic*.]
    \item Anträge zur Tagesordnung sind mindestens eine Woche vor der Mitgliederversammlung beim Vereinsvorstand in Textform einzureichen.
    \item Über die Behandlung von Dringlichkeits- und Initiativanträgen entscheidet die Mitgliederversammlung.
\end{enumerate}

\subsection*{§ 6c Versammlungsleitung}
\begin{enumerate}[label=\arabic*.]
    \item Der Vorsitz des Vorstandes leitet die Versammlung, bei Verhinderung seine Vertretung. Ist auch diese verhindert, bestimmt die Versammlung eine Versammlungsleitung.
    \item Eine Ordnung kann eine von Absatz 1 abweichende Leitung vorsehen.
    \item Die Versammlungsleitung kann abgetreten werden.
\end{enumerate}

\subsection*{§ 6d Beschlussfähigkeit}
\begin{enumerate}[label=\arabic*.]
    \item Die Mitgliederversammlung ist beschlussfähig, wenn mindestens dreißig Prozent aller stimmberechtigten Mitglieder anwesend sind.
    \item Die Beschlussfähigkeit wird auf Antrag unverzüglich festgestellt.
    \item Die Versammlung gilt solange als beschlussfähig, bis das Fehlen der Beschlussfähigkeit festgestellt worden ist.
    \item Abweichend von Absatz 3 ist vor der Abstimmung über in § 6g Absatz 1 aufgeführte Beschlüsse die Beschlussfähigkeit festzustellen.
    \item Ist die Mitgliederversammlung aufgrund mangelnder Anzahl an Teilnehmenden nicht beschlussfähig, ist die darauf folgende ordentlich oder außerordentlich einberufene Mitgliederversammlung ungeachtet der Anzahl an Teilnehmenden beschlussfähig.
\end{enumerate}

\subsection*{§ 6e Beschlussfassung}
\begin{enumerate}[label=\arabic*.]
    \item Auf Antrag eines stimmberechtigten Mitglieds ist geheim abzustimmen.
    \item Die Beschlüsse der Mitgliederversammlung sind zu protokollieren. Das Protokoll ist von der Versammlungsleitung und der Protokollführung zu unterzeichnen und allen Mitgliedern zugänglich zu machen.
    \item Der Widerspruch gegen Versammlungsbeschlüsse oder die Rüge des Protokolls hat unverzüglich, jedoch spätestens mit einer Frist von zwei Wochen nach Bekanntgabe zu erfolgen.
\end{enumerate}

\subsection*{§ 6f Stimmrecht}
\begin{enumerate}[label=\arabic*.]
    \item Jedes ordentliche Mitglied hat eine Stimme.
    \item Das Stimmrecht kann auf andere stimmberechtigte Mitglieder übertragen werden. Das bevollmächtigte Mitglied muss eine schriftliche Stimmrechtsvollmacht vorlegen. Kein stimmberechtigtes Mitglied darf mehr als zwei Stimmrechtsvollmachten auf sich vereinigen.
    \item Fördermitglieder sind nicht stimmberechtigt.
\end{enumerate}

\subsection*{§ 6g Beschlussquoten}
\begin{enumerate}[label=\arabic*.]
    \item Beschlüsse über 
    \begin{enumerate}[label=\roman*.]
        \item Satzungsänderungen (einschließlich der Änderung des Vereinszweckes),
        \item die Abwahl des Vorstandes, oder
        \item die Auflösung des Vereins
    \end{enumerate}
    benötigen zu ihrer Rechtswirksamkeit die Dreiviertelmehrheit der Stimmen.
    \item Beschlüsse über Vereinsordnungen oder Abstimmungen für Ausschlüsse, nach Anfechtung, bedürfen einer Zweidrittelmehrheit der Stimmen.
    \item In allen anderen Fällen genügt die einfache Mehrheit der Stimmen. Bei Stimmengleichheit gilt ein Antrag als abgelehnt. 
    \item Das Stimmenverhältnis bezieht sich stets auf die Anzahl der durch die anwesenden stimmberechtigten Mitglieder, einschließlich der durch sie vertretenen Mitglieder abgegebenen Stimmen ohne Enthaltungen.
\end{enumerate}

\subsection*{§ 6h Wahlen}
Wahlen, wie insbesondere die Wahl des Vorstandes und der Kassenprüfenden, richten sich nach der Wahlordnung.

\section*{§ 7 Vorstand}
\begin{enumerate}[label=\arabic*.]
    \item Der Vorstand besteht aus folgenden Ämtern: 
    \begin{enumerate}[label=\roman*.]
        \item dem oder der Vorsitzenden,
        \item ein bis zwei stellvertretenden Vorsitzenden,
        \item einem Finanzvorstand,
    \end{enumerate}
    \item Der Vorstandsvorsitz, die stellvertretenden Vorsitzenden und der Finanzvorstand bilden den geschäftsführenden Vorstand.
    \item Die Vorstandsmitglieder sind ehrenamtlich tätig, sie haben Anspruch auf Erstattung notwendiger und verhältnismäßiger Auslagen.
    \item Besteht der Vorstand aus weniger als zwei Mitgliedern, so sind unverzüglich Nachwahlen durchzuführen.
\end{enumerate}

\subsection*{§ 7a Besetzung des Vorstandes}
\begin{enumerate}[label=\arabic*.]
    \item Die Amtsdauer der Vorstandsmitglieder beträgt zwei Jahre. Wiederwahl ist zulässig. 
    \item Der Vorstand bleibt bis zur Neuwahl im Amt, damit auch nach Ablauf der Amtsdauer eine ordnungsgemäße gesetzliche Vertretung gesichert ist.
    \item Der Vorstand besteht aus natürlichen Personen, den Vorstandsmitgliedern. Die Bekleidung mehrerer Ämter durch eine Person ist zulässig, sofern dies auf Beschluss der Mitgliederversammlung erfolgt.
    \item Jedes Vorstandsmitglied besitzt in Abstimmungen und Beschlussfassungen des Vorstands eine Stimme, unbeschadet der Anzahl der ausgeübten Ämter.
    \item Sind zwei oder mehr der durch die Mitgliederversammlung gewählten Vorstandsmitglieder dauerhaft an der Ausübung ihres Amtes gehindert, so sind unverzüglich Nachwahlen anzuberaumen.
    \item Die Abwahl eines Vorstandsmitglieds ist nur möglich, sofern zeitgleich eine Neubesetzung des freigewordenen Amtes gemäß dieser Satzung erfolgt.
    \item Alle Vorstandsmitglieder müssen zum Zeitpunkt des Amtsantritts das achtzehnte Lebensjahr vollendet haben.
    \item Vorstandsposten können nur durch ordentliche Mitglieder bekleidet werden.
\end{enumerate}

\subsection*{§ 7b Vertretungsmacht}
\begin{enumerate}[label=\arabic*.]
    \item Vorstand im Sinne des § 26 BGB ist jedes Mitglied des geschäftsführenden Vorstands.
    \item Die geschäftsführenden Vorstandsmitglieder sind alleinvertretungsberechtigt. Ausgenommen sind 
    \begin{enumerate}[label=\roman*.]
        \item Einstellung und Entlassung von Angestellten, 
        \item gerichtliche Vertretung sowie Anzeigen, 
        \item Aufnahme von Krediten, 
        \item Erwerb, Belastung und Veräußerung von Grundstücken und Immobilien,
        \item Gründung, Erwerb und Veräußerung von Gesellschaften sowie Geschäftsanteilen von Gesellschaften zur Verwirklichung der satzungsgemäßen Ziele.
    \end{enumerate}
    Dies ist im Vereinsregister eintragen zu lassen.
    In diesen Fällen wird der Verein durch alle geschäftsführenden Vorstandsmitglieder gemeinsam vertreten.
    \item Der geschäftsführende Vorstand vertritt den Verein nach außen und führt die laufenden Geschäfte des Vereins. Ihm obliegen die Verwaltung des Vereinsvermögens und die Ausführung der Vereinsbeschlüsse.
\end{enumerate}

\subsection*{§ 7c Beschlussfassung des Vorstandes}
\begin{enumerate}[label=\arabic*.]
    \item Zu den Vorstandssitzungen lädt der Vorsitz ein. Die Einberufung hat zu erfolgen, wenn mindestens ein Vorstandsmitglied dies in Textform verlangt.
    \item Der Vorstand ist beschlussfähig, wenn die Mehrheit der Vorstandsmitglieder anwesend ist.
    \item Beschlüsse werden mit einfacher Mehrheit der anwesenden Mitglieder gefasst. Bei Stimmengleichheit gilt ein Antrag als abgelehnt.
    \item Beschlüsse können auch im Umlaufverfahren, fernmündlich, telegrafisch, fernschriftlich, im Rahmen von Netzkonferenzen oder ähnlichem gefasst werden.
    \item Über Beschlüsse des Vorstandes ist stets ein Beschlussprotokoll anzufertigen und allen Vorstandsmitgliedern in Textform zuzusenden. 
\end{enumerate}

\subsection*{§ 7d Kassenprüfung}
\begin{enumerate}[label=\arabic*.]
    \item Die Mitgliederversammlung wählt ein bis zwei Kassenprüfende, die nicht Vorstandsmitglieder sind, auf die Dauer von zwei Jahren. Wiederwahl ist zulässig.
    \item Die Kassenprüfung erfolgt nach Ablauf eines jeden Geschäftsjahres und prüft die rechnerische Richtigkeit der Buch- und Kassenführung. 
    \item Nach Durchführung ihrer Prüfung informieren die Kassenprüfenden den Vorstand von ihrem Prüfungsergebnis.
    \item Die Kassenprüfenden erstatten Bericht in der nächstfolgenden ordentlichen Mitgliederversammlung.
\end{enumerate}

\section*{§ 8 Ordnungen}
\begin{enumerate}[label=\arabic*.]
    \item Vereinsordnungen dürfen - insbesondere zur Gründung, Führung und Auflösung von Abteilungen, zur Regelung der Durchführung von Versammlungen, Sitzungen und Tagungen der Organe des Vereins und seiner Abteilungen, der Rechte und Pflichten der Mitglieder, der Vereinsfinanzen und der Beiträge - erlassen werden.
    \item Die Vereinsordnungen sind nicht Satzungsbestandteil und dürfen der Satzung nicht widersprechen. Im Zweifel gelten die Regelungen der Satzung.
    \item Vereinsordnungen werden von der Mitgliederversammlung erlassen, geändert oder aufgehoben.
\end{enumerate}

\section*{§ 9 Übergang des Vereinsvermögens}
\begin{enumerate}[label=\arabic*.]
    \item Bei Auflösung oder Aufhebung des Vereins fällt das Vermögen des Vereins an eine von der Mitgliederversammlung zu bestimmende juristische Person des öffentlichen Rechts oder eine steuerbegünstigte Körperschaft.
    \item Der Beschluss hat bei Auflösungs- bzw. Aufhebungsbeschluss in Einheit mit diesem zu erfolgen.
    \item Der Beschluss ist bis zur Einwilligung des Finanzamtes nur schwebend wirksam und darf erst nach erfolgter Einwilligung ausgeführt werden.
\end{enumerate}

\end{document}